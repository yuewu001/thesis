\chapter{国内外研究现状和工作基础}
随着智能手机及其他移动设备的普及,
社交多媒体数据的规模也呈现出了爆发式的增长。
对于社交多媒体数据的挖掘和利用已经成为了当前研究的热点问题,
相关的成果既有利于推动计算机视觉以及多媒体领域相关课题的创新,
对于用户体验的提升以及工业界的发展都也具有重要的应用价值和现实意义。

本文主要研究社交多媒体数据的包括包括语义理解和关联表达两个方面,涉及
弱监督深度学习,特征提取,模型简化,照片集的关联表达和移动多摄像头视频自动剪辑
几个关键问题。本章内容对这些关键问题的研究现状和工作基础做详细回顾。
首先回顾弱监督学习,总结近年来弱监督分类和弱监督深度神经网络的发展情况;
然后针对社交多媒体数据的特征提取, 回顾传统的批处理方法,
解决大规模流数据的在线学习方法,和在线特征提取算法,以及他们在
解决大规模社交多媒体数据特征提取问题中的问题和不足;
在模型简化方面,主要介绍与深度卷积神经网络相关的模型简化工作;
在关联表达方面,从照片集和移动多摄像头视频两个角度分别介绍学术界和工业界
的研究成果和代表性应用产品,指出他们在社交多媒体数据关联表达中
的不足,引出本文提出的照片集关联表达系统和移动多摄像头自动剪辑系统。

\section{弱监督学习}
弱监督学习是指数据标注不完备或者包含噪音条件下的分类问题。本文主要针对
弱监督图像分类问题做工作总结和算法创新。具体的方法可以分为两个方面:
数据去噪(Data Cleaning)和鲁棒噪音分类模型(Noise Robust Models)。

\subsection{数据去噪}
\label{sec:data-cleaning}
数据去噪是指找出并移除可能错误标注的数据。
数据去噪的优点在于它不依赖于目标任务的模型和训练方法。
但是数据去噪也面临区别噪音数据和异常数据的难题~\cite{danyluk2014small}。
数据去噪方法会导致两种类型的错误:正确标注的样本被误判为错误标注并被丢弃;
错误标注的样本被漏判。

经典的数据去噪方法只基于图片本身的视觉特征。
Brodley等人提出的交叉过滤法~\cite{brodley1999identifying}采用类似于交叉验证的思路,
将训练数据分成$n$等份, 并选择$m$种分类算法(称为过滤算法)。
对于每份数据,在剩下的$n-1$份数据上训练$m$个分类模型。
然后用得到的$m$个分类器对这部分数据进行预测,
最后通过一定的过滤算法判定错误标注的数据并将之移除。
最大边界难分类样本学习通过迭代地学习正负类别分界面,
保留难分类样本,移除易分类数据,从而更新正负样本集的方法
来达到去除噪音数据的目的~\cite{zhou2015conceptlearner}。

交叉过滤法和最大边界难分类样本学习算法直接利用弱监督的数据标注,属于
有监督学习方法。研究人员也提出了一些半监督的噪音数据去除方法。半监督学习方法
首先需要人工标注部分数据作为种子数据,记为$\L = \{(\x_i, y_i)\}_{i=1}^m$。剩下
的数据记为$\U = \{\z_j\}_{j=1}^n$。通常情况下,$m \ll n$。
核平均算法的核心思想是通过对未标注的数据进行加权,
使得加权后的数据分布与有标注数据的分布相同~\cite{huang2006correcting,vo2015deep}。
直推式支持向量机(Transductive Support Vector Machine, TSVM)
~\cite{sindhwani2006large,vo2015deep}
同时根据有标注的数据$\L$和未标注的数据$\U$寻找分界面,
并约束最多有$r$个未标注数据被判定为正类样本。

此外,由于社交多媒体数据具有丰富的上下文(context)信息,
这些上下文信息也从侧面反映了图片的语义内容, 因此可以用来辅助数据去噪。
Schroff等人提出了一种基于图片在网页中的文本上下文信息以及
图片本身的视觉特征对图片进行重排序的算法~\cite{schroff2011harvesting}。
算法首先利用图片附近的文本上下文学习标注是否正确的后验概率并进行重排序。
然后从基于文本的重排序中选择前$n_+$个图片作为正样本,
再从所有其他类别的图片中选择$n_-$个图片作为负样本。对图片提取视觉特征以后,
训练一个SVM分类器。社交主动学习(Social Active Learning)首先用视觉特征训练
分类器,并评价数据对于提升分类效果的信息量,同时利用文本特征评价每个数据标注可靠
的置信度,综合考虑信息量和置信度选取数据,
迭代地训练模型并选取正确标注的数据~\cite{chatzilarisalic}。

可以看出,数据去噪方法分成两步进行:特征提取和数据去噪的模型训练,
对于噪声的判别也依赖于人为假设和定义的准则,这些因素导致在真实的
社交多媒体数据上,数据去噪方法的效果受多重因素影响,很难达到理想的效果。因此,
我们需要提出一些端到端的学习方法(End-to-End Learning),并利用数据之间的关联
达到弱监督学习的目的。

\subsection{鲁棒噪音模型}
鲁棒噪音模型是指本身对噪音数据不敏感或具有抑制作用的模型。
在经验风险最小化(ERM)规则和给定的损失函数下,
如果模型发生分类错误的概率保持不变并且与标注噪音无关,
则认为模型是对标注噪音鲁棒的。研究人员讨论了在特定条件下,理论上是否存在
对标注噪音完全鲁棒的模型~\cite{manwani2013noise}。 例如,
$0-1$损失函数在均匀的标注噪音或者能达到$0$错误率的情况下是
噪音鲁棒的~\cite{thathachar2011networks,sastry2010team}。
Beigman等人讨论了非随机噪音模型下的鲁棒模型~\cite{beigman2009learning},
最小平方差损失函数在均匀标注噪音下同样是噪音鲁棒的。其他常见的损失函数,如
指数损失函数,对数损失函数,以及Hinge Loss均不是噪音鲁棒的。换句话说,大部分常见的
机器学习算法都不是完全噪音鲁棒的,但可以在一定程度上减小错误标注的影响, 提升
模型的鲁棒性。

对于经典的SVM分类器,Bunescu等人提出了稀疏多实例学习算法用于解决训练样本中的
噪音数据问题~\cite{bunescu2007multiple,vijayanarasimhan2008keywords}。
多实例学习将一组包含正样本包的有噪音数据称为一个正样本包,将一组负样本数据称为负样本包,同时假设正样本包
中至少包含一个真实的正样本,负样本包中全是负样本。通过对正负样本包采用
不同的约束条件和惩罚稀疏,达到抑制噪音影响的目的。

逻辑回归是一种经典的概率统计分类模型。 相比于其他分类模型(如SVM),
逻辑回归的能够输出样本是否为某个类别的概率,因此得到了十分广泛的应用。
然而,逻辑回归算法对于训练数据中的噪音标注十分敏感。
Feng等人提出了一种基于次高斯分布的鲁棒逻辑回归算法~\cite{feng2014robust}。
根据理论分析结果,算法首先去除范数大于一定阈值的数据,
在剩余的数据中仅最大化前$n_1$数据的标注和预测结果之间的相关度。
在随机噪音(Noise at Random, NAR)假设下,研究人员提出用隐藏变量对数据的真实类别,错误标注的转移概率
以及他们与观测到的标注之间的关系进行建模,
并通过EM算法优化目标结果~\cite{izadinia2014image,izadinia2015deep}。

近年来,深度卷积神经网络在图片识别上获得了巨大的成功。
卷积神经网络最早于1998年被Lecun等人运用在了文字识别上~\cite{lecun1998gradient}。
随着显卡能力的提升,,
得益于显卡设备的快速发展,卷积神经网络的深度、宽度以及结构不断改进,
深度卷积神经网络的识别能力和应用范围都得到了巨大的
提升~\cite{krizhevsky2012imagenet,simonyan2014very,szegedy2015going,szegedy2016rethinking,szegedy2016inception,he2016deep}。

然而,深度卷积神经网络需要大量的训练数据,对于数据标注的质量有很大的依赖性。
如何利用容易获得的大量弱监督数据训练深度卷积神经网络成为了近年来的研究热点之一。
自举深度神经网路是2014年谷歌研究团队提出的利用图片特征之间的相似性
监督网路学习,抑制错误标注影响的学习方法~\cite{reed2014training}。文章
认为,如果图片的特征之间具有相似性,那么预测的结果同样也应该比较类似,
并称之为感知连续性。在随机噪声的假设下,在神经网络中加入全连接隐藏层表示真实的
类别,为了达到感知连续性的约束,论文提出引入类自适应编码器的方法训练网络。此外,还可以
通过约束隐藏层输出的熵最小达到软自举优化的目的,
以及约束隐藏层概率最大类别的熵最小达到硬自举的目的。Sukhbaatar等人则通过约束
隐藏层参数的迹来达到感知连续性的优化目标~\cite{sukhbaatar2014training}。
Xiao等人则综合考虑了标注可能遇到的随机噪音和非随机噪音,通过隐藏变量对不同噪音
类型下的概率进行建模,用两个神经网络分别学习噪音类型和真实的标注,通过EM算法
学习整个模型的参数~\cite{xiao2015learning}。
以上方法都基于特定的噪音模型, Azadi等人提出
一种辅助图片正则项(Auxiliary Image Regularizer, AIR)的方法~\cite{azadi2015auxiliary},
根据数据的特征结构,通过组约束的方法使得只有部分数据具有响应,
从而在训练数据中识别出有用的辅助数据,更好地训练神经网络。某种程度上,
可以认为辅助图片正则项方法是在训练数据中寻找最近邻数据,
减少深度模型对噪音数据的拟合。

传统的鲁棒噪音模型方法仍然基于特征提取和模型训练两个步骤,
具有章节~\ref{sec:data-cleaning}提到的局限性。现有的弱监督深度学习
方法基于特定的噪音模型,难以处理真实的场景,或建模过于复杂,难以
训练。AIR方法利用图片之间的相似性抑制噪音,但组稀疏优化提高了
网络训练的难度和时间开销。因此,我们提出一种新的利用数据在特征空间的关联性
进行相关反馈的弱监督深度神经网络,并对网络进行简化和近似,提高模型的
适用性和实用性。

\section{特征选取}
图像特征表述不仅包括高层次的卷积神经网络特征,
还包含低层次全局特征(如颜色特征~\cite{jain1996image},
边缘特征~\cite{jain1996image} ,纹理特征~\cite{manjunath1996texture}),
局部特征(如SIFT~\cite{lowe1999object} ,SURF~\cite{bay2006surf})
以及用来通过局部特征描述整体视觉信息的词袋方法等~\cite{yang2007evaluating}。
实际应用中需要根据需求选取对目标任务最有用的特征子集,
这对于计算能力,内存,和电量都十分受限的移动设备尤其重要。
此外,去除特定任务不相干的特征,还可以提高特征的表达能力。
特征选取在机器学习和数据挖掘领域得到了广泛的研究,
可以分成两类:批处理方法和在线特征选取。

\subsection{批处理方法}
批处理方法是指每次迭代都需要考虑所有的训练数据,可以分为三个类别:
\begin{itemize}
    \item 过滤法: 过滤法分析特征之间的关联,距离,交互信息熵等,
        选取最有代表意义的特征子集~\cite{yu2003feature,jiang2015relative,li2016feature} 。
        Yang等通过分析指出,传统的过滤法存在单调性的问题,
        不同大小的特征子集之间存在单调的包含关系,
        这种包含关系在实际情况中并不成立~\cite{yang2013efficient}。
        他们对特征之间的联系进行建模,提出了一种多核学习的方法。
    \item 包装法: 包装法使用预先定义好的分类器去评价
        选取的特征子集的性能~\cite{kohavi1997wrappers}。
        这类方法迭代的选取不同的子集,并得到该子集在对应分类器上评价指标,
        虽然能够获得该分类器上最好的特征子集,但是计算的开销也十分巨大,
        因此对于该类方法的研究相对较少。
    \item 嵌入法: 嵌入法将特征选取与模型训练进行融合,
        是一种在高效的过滤法和高准确率的包装法之间
        综合平衡的方案~\cite{pappu2015sparse,le2014feature}。
\end{itemize}

\subsection{在线特征选取}
批处理方法的缺点在于需要将所有训练数据都加载到内存中,
对于目前的大量高维数据,这类方法的局限性十分明显。
同时,批处理方法要求数据预先已经全部存在,实际场景中,存在大量的流媒体数据。
因而近年来,随着数据量的增大和维度的增加,大量的工作转向了在线学习。
最早的在线学习算法是1958年提出的感知机算法。
2006年,Crammer等人在感知机算法上增加约束,
使得模型在每次迭代后在当前数据上都能获得正确的分类结果~\cite{crammer2006online}。
考虑到批处理学习算法中,二阶海森矩阵能够显著提高算法的收敛速度,
Crammer等人假设模型参数服从一个高斯分布,用协方差矩阵表示当前模型对于参数的不确定性,
提出了置信度加权的在线学习算法CW~\cite{crammer2009multi}。
该算法每次迭代时约束更新后的模型以一定概率在当前的数据上获得正确的预测结果。
由于该算法假设数据的标注都是正确的,在实际场景中的效果会收到影响。
自适应的置信度加权在线学习算法则降低了对于噪音数据的敏感性~\cite{crammer2009adaptive}。

近几年,在线学习被应用到特征选取上。Langford等人提出的稀疏在线学习算法,
在感知机模型上增加了模型参数的L1范数作为正则项,
获得稀疏的模型~\cite{langford2009sparse}。
Duchi等提出的FOBOS算法将稀疏在线学习分成两步,第一步是正常的在线学习,
第二步优化目标使得模型尽可能接近第一步得到的参数,
同时对应的L1范数最小~\cite{duchi2009efficient}。
另一种思路则是优化模型在主空间和对偶空间的距离,
利用模型的L1范数达到稀疏模型的目的而提出的RDA算法~\cite{xiao2010dual} 。
RDA算法在高稀疏条件下往往能获得更好的特征表达能力。
受置信度加权等二阶算法的启发,Duchi等利用梯度的协方差构建二阶信息,
提出了自适应的二阶FOBOS算法和RDA算法。

基于L1范数的稀疏在线学习算法是针对特征选取的一种“软”约束。
参数设定与目标特征数目之间没有确定的关联。
基于L0范数的在线特征选取也得到了研究人员的关注。
Wu等提出的在线特征流学习能够在每次迭代后返回一个模型和它选取的特征子集~\cite{wu2010online}。
该算法假设每次获得所有数据的某个特征,不同特征按照时序被送到算法中。
另一种更普遍的应用场景下,算法每次可以获取一个数据的部分或者所有特征,
不同数据按照时序到达算法。Huang等提出一种无监督的在线特征选取算法处理这类数据~\cite{huang2015unsupervised}。
Wang等则利用有监督的数据,根据权重向量的绝对值进行在线特征选取~\cite{wang2014online}。

批处理方法的主要问题在于不具有可伸缩性(Scalability),以及对于流数据不具有
很好的应对能力。基于L1范式的在线稀疏学习方法对于特征的数目不能做直接的约束,
在实际应用中需要根据不同的数据调整参数达到选取预期特征数目的目的。当前的在线
特征选取算法根据权重向量绝对值选取特征的做法与批处理方法的效果还有较大差距,并且
算法的复杂度较高。因此,本论文提出了大规模高维特征选取算法,不仅具有与批处理方法
相近的准确率,也极大地减小了计算复杂度,对于处理大规模社交多媒体数据具有非常大
的应用价值。

\section{模型简化}
近年来,深度卷积神经网络在图片识别,物体检测等领域获得了巨大的成功,
为了进一步提高网络的表征能力,研究人员不断改进网络的
深度~\cite{simonyan2014very},
宽度~\cite{zagoruyko2016wide}以及
拓扑结构~\cite{szegedy2015going,srivastava2015highway,he2016deep}。
然而,大量的网络参数也要求大量的时间开销和计算资源,也极大地限制了
深度网络在计算能力、存储空间和电池续航受限的移动设备上的应用。如何在
不影响网络性能的情况下减少网络参数成为了当前的研究热点问题。深度网络模型简化
相关的工作可以分为三个类别:矩阵分解、量化以及稀疏优化。

矩阵分解利利用参数之间的相关性,对参数矩阵进行低秩分解,从而减少网络的参数个数。
Denil等人提出将参数矩阵分解成两个低秩矩阵的乘积,将其中一个矩阵作为特征空间的一组基,
并提出了基向量字典的构建方法~\cite{denil2013predicting}。
Denton等人提出在网络的预测阶段,用奇异值分解(Singular Vector Decomposition, SVD)
对参数矩阵进行分解,如果参数矩阵的奇异值迅速下降,则参数矩阵能够被
前最大$t$个奇异值及对应的奇异向量很好地近似~\cite{denton2014exploiting}。
Rigamonti等人提出的方法在空间上对每个通道的卷积核用秩为$1$的矩阵进行近似, 从而减小
计算量~\cite{rigamonti2013learning},Jaderberg等人在空间分解的基础上进一步利用
通道之间的冗余信息,将原始的卷积操作分解成两步卷积运算~\cite{jaderberg2014speeding}。
Ioannou等人和Tai等人进一步改进并扩展低秩分解方法,
将其用于更大的深度网络~\cite{ioannou2015training,tai2015convolutional}。
Mamalet等人将卷积核分解为秩为1的向量乘积,并与后续的池化(Pooling)操作冗合为
一层卷积运算,从而减少运算量~\cite{mamalet2012simplifying}。

量化是指利用较少的比特数表示网路参数,从而减少模型的大小以及乘法运算的复杂度。
目前常用的网络参数都采用32比特的浮点型数据,研究表明,
可以利用更少的比特数来表示每个参数。例如,Hwang等人和Arora等人提出仅用$+1, -1,
0$三个数值表示网络参数训练卷积神经网络~\cite{hwang2014fixed,arora2014provable}。
Courbariaux等人和Rastegari等人则进一步提出用二个数值
表示网路参数~\cite{courbariaux2015binaryconnect,rastegari2016xnor}。
Gong等人提出用向量量化的方法对全连接层的参数进行量化~\cite{gong2014compressing}。
针对卷积层的向量量化则在Wu等人提出的Q-CNN得到研究和应用~\cite{wu2016quantized}。
Anwar等人用最小平方差的方法量化网络~\cite{anwar2015fixed}。
Chen等人利用哈希函数随机将网络参数分组,达到量化的目的~\cite{chen2015compressing}。

当前网络的参数矩阵是密集矩阵, 稀疏优化的目标是使得最终的参数矩阵稀疏,
从而达到模型简化的目的。
区别于参数矩阵低秩分解,Liu等人提出对卷积核进行稀疏分解,并提出了高效的系数矩阵
相乘算法~\cite{liu2015sparse}。受L1或L2约束的启发,Han等人提出重复交替进行
删除神经元之间连接和重新训练精简后网络~\cite{han2015learning,han2015deep}。
然而,这些稀疏方法产生的稀疏性不是结构化的,运算时会导致无规则的内存访问,
不能带来实际的运算加速。Li等人根据卷积核的绝对值之和去除部分卷积核,从而
达到运算的加速~\cite{li2016pruning}。
Murray和Chiang运用结构化稀疏方法约束隐藏层神经元的个数~\cite{murray2015auto}。
Anwar等提出了卷积核,通道,以及卷积核内部的结构化稀疏方法~\cite{anwar2015structured}。
他们还提出用粒子滤波器(Particle Filter)衡量网络连接的重要性,从而
优化网络结构。Wen等人系统讨论了结构化的稀疏算法,从卷积核,通道,卷积核形状,
深度四个方面对网络进行进行结构化的约束,
不仅达到了减少网络参数的目的,还获得了实际运算速度上的提升。
Hu等人通过研究发现大网络的部分神经元的响应大部分情况下为0,且与网络的
输入信号无关。因此,他们通过分析网络神经元在大数据集上的响应去除部分
神经元~\cite{hu2016network}。
这些结构化的方法稀疏后的网络,有连接的神经元之间在通道上仍然是密集连接,
Soravit等人提出了对通道稀疏连接的方法,在保持其他结构化方法同样计算
速度的情况下获得了很好的效果~\cite{changpinyo2017power}。

以上介绍的模型简化方法,虽然取得了一定的效果,但同时也存在很多的问题。
矩阵分解方法对于全连接层以及大卷积操作具有非常好的效果,然而最新的网络
更倾向于使用更少的全连接层,并通过级联小卷积核的方法达到大卷积核同样的
感知野,不仅减少了运算量,还提高了网络的表征能力~\cite{szegedy2016rethinking}。
量化方法需要特定的硬件或软件库的支持才能显著提高运算的速度。
非结构化的稀疏优化方法对于减少参数数目作用比较明显,对于计算速度的提升十分有限。
结构化的方法一般基于参数的绝对值决定参数的重要性,具有一阶在线学习方法同样的
缺陷,而组稀疏优化的方法则增加了网络优化的难度。为此,有必要提出一种
新的模型简化方法,既能保证简化后网络的表征能力,实际提升网络的运算效率,
减少参数规模,又易于优化,提高模型简化的可操作性。

\section{社交多媒体数据的关联表达}
本节从照片集关联表达和视频关联表达两个方面回顾社交多媒体数据关联表达相关的工作。

\subsection{照片集关联表达}

照片集的关联表达涉及事件检测(Event Detection),关键照片选取,
和照片故事表达。通常,用户照片包含拍摄时的时间戳和位置信息,
我们可以利用这些信息将照片集分成不同的事件。Platt等人提出用
一个小时或者自适应的时间间隔作为相邻事件之间的时间间隔~\cite{platt2003phototoc}。
Graham等人扩展了该方法,使用事件聚类的类内内拍照频率和
类间时间间隔调整已有的事件划分~\cite{graham2002time}。
Gargi提出将拍摄频率的急速增加的时间点作为时间的起点,将长时间间隔没有拍摄
作为事件的终点~\cite{gargi2003modeling}。Matthew等人将可信度,动态规划
或者贝叶斯信息准则(Bayes Information Criterion, BIC)运用到照片的
相似度矩阵,从而检测事件的边缘位置~\cite{cooper2005temporal}。
一般来说,事件检测问题可以表示为一个聚类问题。Loui和Svakis提出用
2类的K-means聚类算法将照片分组,并检查照片之间颜色的相似性改进
聚类~\cite{loui2000automatic}。Gong等人利用层次聚合聚类算法将照片
分配到不同的聚类中心~\cite{gong2007segmenting}。
Platt等人用隐马尔科夫模型聚类~\cite{platt2003phototoc}。
Mei等人在时间,位置以及内容特征上利用混合高斯模型解决事件检测问题~\cite{mei2006probabilistic}。
Xu等人进一步利用纹理和深度特征改进了这个算法~\cite{shen2016multi}。

近年来许多研究工作和产品相继出现,用以解决关键照片选取问题。在学术界,
关键照片选取主要依赖照片的代表性~\cite{cooper2005temporal,mei2006probabilistic,shen2016multi,chu2008automatic}。
Cooper等人将事件中第一张照片作为关键照片。Mei
等人选择具有最大后验概率的照片~\cite{mei2006probabilistic}。
Chu等人提出在照片的聚类中,根据近似图片对之间的相互关系决定关键照片~\cite{chu2008automatic}。
Xu等人则根据事件的重要性引入了照片的受欢迎程度(popularity)以及时间内部的相似度决定
关键图片~\cite{shen2016multi}。

照片故事表达一直以来受到了工业界和学术界共同的关注。
例如,Magisto\footnote{\url{http://magisto.com}}和
Animoto\footnote{\url{http://animoto.com}}是两个可以根据用户提供的照片产生
音乐视频(Music Video)的在线服务。然而,它们依赖用户主动选取和提供的照片,
不能直接从照片集中总结并整理出故事呈现给用户。此外,用户需要手动指定
音乐视频编辑的风格。其他的在线服务,如Microsoft
Onedrive\footnote{\url{https://onedrive.live.com}},
Google\footnote{\url{https://plus.google.com}}可以在一定程度上对照片集进行
事件检测和照片选取,并且缺乏对数据的重新表达。
在学术界,Hua等人提出的Photo2Video系统是从照片产生视频
的先驱性工作~\cite{hua2006photo2video},利用相继运动将静态照片转换成运动片段,
最终通过转场效果并与节奏匹配生成最终视频。
然而该系统没有设计不同的编辑风格,所采用的编辑效果也比较单一, 其他系统比如
Tiling SlideShow系统将照片和背景音乐同步,
并以瓷砖式幻灯片的方法播放~\cite{chu2007tiling}。
Kuo等人提出的Sewing Photos系统专注于解决在播放照片幻灯片时,
给照片之间分配平滑的转场效果~\cite{kuo2011sewing}。
Sewing Photos和
Tiling SlideShow也存在Photo2Video同样的编辑效果单一的问题。
Yang等人提出了一种从照片中自动生成有吸引力的版面设计的算法。

以上系统没有系统对照片集进行总结整理,在表达时
很少运用丰富的视频制作特效和视频编辑风格和编辑语法,因而对于照片故事
的表现能力十分有限。因此,我们需要提出一个能够对照片集进行事件挖掘和关键图片选取,
并运用专业的编辑语法对故事进行再现表达,提供更为有效的照片集关联表达方法。
在表格~\ref{tab:monet-comp}中,我们比较了现有照片集关联表达系统和本文提出的Monet系统之间的差异。

\begin{table}[htbp]
    \centering
    \caption{照片集关联表达系统比较} \label{tab:monet-comp}
    \begin{tabular}{|c|c|c|c|c|}
    \hline
     & Magisto  & Animoto &  Google+ & Monet (this paper)\\
    \hline
    相机运动分析 & + & - & - & +  \\
    视频分析    & - & - & - & + \\
    人脸检测识别 & + & - & + & + \\
    场景分析 & + & - & + & + \\
    物体识别 & - & - & + & + \\
    音乐分析 & + & - & - & + \\
    照片分组和选取 & - & - & + & + \\
    设计风格 & + & - & - & + \\
    色彩调整 & + & + & + & + \\
    社交和云存储 & - & - & + & +  \\
    \hline
\end{tabular}
\end{table}
\subsection{移动多摄像头视频关联表达}
移动多摄像头视频是指在同一个事件中,有多个移动摄像头从多个角度拍摄的,时间上有
重叠的一组视频~\cite{DBLP:conf/mm/ShresthaWWBA10}。
随着智能设备的普及与性能的提升,移动多摄像头视频的自动剪辑成为了近年来
的热点问题。

移动多摄像头视频的自动剪辑的方法可以总结为三个类别:
基于规则的~\cite{DBLP:books/daglib/0023820},
基于优化的~\cite{DBLP:conf/mm/ShresthaWWBA10},
和基于学习的~\cite{DBLP:conf/mm/NguyenSNO13,DBLP:conf/mm/SainiGYO12}。
基于规则的方法模仿专业视频人员的编辑过程。然而,移动多摄像头视频的自动剪辑
过程更类似于用户的选择倾向,而并非固定的编辑规则。Shrestha等人提出了从
视频质量,多样性和切换点的合适度进行镜头选取的优化算法~\cite{DBLP:conf/mm/ShresthaWWBA10}。
然而实际系统中,论文并没有提出切实可行的切换点合适度的评价方法,仅仅
考虑了视频质量和多样性。在视频质量评估中,没有考虑到移动视频中的倾斜和遮挡问题。
此外,作者通过贪心算法解优化方程,仅获得了目标方程的局部最优解。
Saini等人提出Jiku Director用于解决在线视频剪辑问题~\cite{DBLP:conf/mm/NguyenSNO13,DBLP:conf/mm/SainiGYO12}。
他们通过学习隐马尔科夫模型(HMM)用于镜头选取和确定镜头长度。
然而,通过这种方式学习到的规则与内容无关,而实际中,镜头角度和长度的
选取都是和内容密切相关的,并且受运动强度,音乐的节奏,
以及其他因素的影响~\cite{DBLP:conf/mm/HuaLZ04a}。此外,该系统由于不能准确判断
视频的角度,因而不能做到全自动的视频剪辑,尤其是在模型的训练阶段,需要人工的干预。
Arev等人最近提出了从多摄像头视频的自动编辑系统~\cite{Arev:2014:AEF:2601097.2601198},
但是该系统十分依赖场景的三维重建,不适用于移动多摄像头。

以上系统均没有考虑音频的剪辑,而完整的视频是有音频流和视频流两部分构成的,
高质量的音频流对于提升用户体验具有十分重要的作用。
在表格~\ref{tab:mashup-comp}中,我们比较了现有移动多摄像头视频系统和本文提出的MoVieUp系统之间的差异。

\begin{table}[htbp]
    \centering
    \caption{视频自动剪辑系统比较} \label{tab:mashup-comp}
    \begin{tabular}{|c|c|c|c|}
    \hline
     & MoVieUp (this paper) & VD~\cite{DBLP:conf/mm/ShresthaWWBA10}
     & Jiku~\cite{DBLP:conf/mm/SainiGYO12}\\
    \hline
    diversity & Yes & Yes & Yes \\
    shakiness & Yes & Yes & Yes \\
    tilt & Yes & No & Yes \\
    occlusion & Yes & No & Yes \\
    audio mashup & Yes & No & No \\
    cut point & Audio+Video & Manual & Learning \\
    \hline
\end{tabular}
\note{VD is short for Virtual
    Director~\cite{DBLP:conf/mm/ShresthaWWBA10}. Transition matrix for cut points learnt by Jiku Director is the same to
    all videos, thus not content-based.}
\end{table}

移动多摄像头自动剪辑还与视频编辑相关,包括视频摘要(Video Summarization),
相机选取,以及家庭或音乐视频编辑。视频摘要与视频剪辑的共同点在于它们都要最大化
有信息内容的部分。Sundaram等人提出了从可计算的镜头中生成快速概览的
实用框架~\cite{sundaram2002computable}。该论文将视觉编辑语法运用到镜头编辑中(选取,
缩放,时长,顺序等)~\cite{CinemaElements1982},对于本文的工作具有很大的借鉴意义。

可计算镜头的检测通常基于人类记忆的一个仿真模型~\cite{sundaram2002computable}。
相机选取在演讲和会议等诸多特定场景都得到了广泛的研究,
通常可以通过识别演讲者或者检测人脸来选取需要展示的相机内容~\cite{DBLP:journals/ieeemm/LampiKBE08,sumec2006multi}。
Ranjan等人和Zhang等人提出的系统中用跟踪和基于音频的定位来选择相机。以上系统都可以
归结为基于音频的方法。在移动多摄像头视频自动剪辑中,音频不是唯一的关注点,音频定位和人脸检测
在嘈杂的拍摄环境以及低视觉质量的条件下的应用能力十分有限。

此外,还有大量关于家庭视频或音乐视频编辑的工作。Hua等人提出了自动家庭视频编辑系统
AVE,从一系列的家庭视频中提取一部分最精彩的镜头~\cite{DBLP:conf/mm/HuaLZ03}。
他们提出了两套规则分别保证对原来视频的代表性,以及音频和视频之间的协调性。
类似的方法被拓展到自动音乐视频编辑,该系统分析视频的时序结构和音频的节奏并进行匹配~\cite{DBLP:conf/mm/HuaLZ04a}。
然而,由于移动多摄像头视频需要进行时间上的同步,并且需要保证内容上的
质量和多样性,这些系统不能直接用于移动多摄像头视频自动编辑。

