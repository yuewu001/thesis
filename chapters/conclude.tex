\chapter{总结与展望}
本文的主要目标是研究并实现社交多媒体数据内容分析和处理系统,从而完成对社交多媒体
数据的语义理解和关联表达。
语义理解的内容包括:
1)解决语义理解标注难的问题:从大规模标注不准确不完备的社交多媒体数据中学习语义理解模型;
2)解决大规模社交多媒体数据处理慢的问题:从大量图片特征中选取对问题有用的特征,加快语义理
解相关问题的处理速度, 以及简化语义理解模型,减小模型参数,加快语义理解的速度。
社交多媒体数据关联表达是指根据用户个性化的需求,从社交多媒体数据中选择有关联的数据,并
以一定的表达形式将这些关联的数据呈现给用户。本论文分别从图片和视频的角度,研究关联表达的具
体应用:1)解决社交多媒体数据中的照片集关联表达问题:选取语义上有关联的代表性图片,通过可计
算的视频编辑语法,对照片集进行故事化表达; 2)解决社交多媒体数据中的移动多摄像头视频自动剪
辑问题:将同一时间同一地点拍摄的多摄像头视频在时间上进行同步,通过可计算的视频编辑语法,选
取镜头和录音,将多摄像头视频剪辑成单一的音视频流。

本章首先对本文工作进行总结,再给出今后研究方向的展望。

\section{本文总结}
本文的主要研究内容是社交多媒体数据的语义理解和关联表达,主要解决社交多媒体数据标注难,处理慢,
以及照片集和移动多摄像头视频自动剪辑的关联表达问题。本文的主要创新点包括:

(1)弱监督社交多媒体数据语义理解

社交多媒体数据的语义理解是处理社交多媒体数据的基础。传统的语义理解方法通常
依赖数据准确地标注信息,而在社交多媒体数据的标注通常包含大量的噪音,影响了已有方法的准确性。
现有的噪音鲁棒算法通常依赖特定的噪音假设或模型以及相应的训练方法过于复杂。
针对这一现状,本文提出了一种新的对噪音鲁棒的相关反馈弱监督深度神经网络,解决
有噪音情况下的任意语义类别的学习问题。该方法基于感知连续性的假设,即语义上接近的数据
在特征空间上也比较接近,从而利用特征之间的相关性使得不同的数据在训练工程中有不同的梯度贡献。
算法将数据特征转化为相似性表示,并利用相似矩阵的低秩近似实现感知连续性。为了减小模型训练的难度,
本文进一步对目标函数进行了近似和简化,提出了高效的相关反馈弱监督神经网络。与传统的语义理解方法
以及弱监督语义理解方法相比,本文提出的相关反馈网络具有更好的噪音鲁棒性。

(2)大规模社交多媒体数据快速处理

社交多媒体数据的快速处理是将语义理解应用到实际问题的关键。社交多媒体数据具有数据规模大,
特征种类多,应用场景复杂多样的特点。不用的应用场景需要不同的特征表述,冗余特征的存在会影响
模型的效率和效果。此外,当前移动端的处理逐渐成为趋势,移动设备的计算能力,存储空间以及电池
容量都十分有限,提高社交多媒体数据的处理速度十分必要。本文首先从特征选取的角度
实现社交多媒体数据的快速处理。现有的批处理特征选取算法存在计算慢,内存占用高,不能处理流数据
的缺点,而已有的在线特征选取算法的效果与批处理算法有明显的差距。本文利用二阶在线学习算法,
基于特征的置信度进行特征选取,并利用最大最小堆结构提出快速特征选取算法,将二阶在线特征选取算法
的复杂度降低成与非零特征数目成正比。在生成数据集与公开数据集上的比较结果验证了本文提出算法的
有效性和高效性。

此外,深度卷积神经网络广泛应用于社交多媒体数据的语义理解。然而深度网络的网络深度和参数个数
都比较大,需要大量的计算资源和时间。现有的深度卷积神经网络简化算法需要依赖特定的硬件或软件库
的支持,或优化困难。本文利用二阶在线特征选取算法提出了模型简化算法,在不影响网络表征能力
的情况下,减少了模型参数,提高了模型速度。

(3)社交多媒体数据的关联表达

社交多媒体数据的关联表达是语义理解的最终目的。当前,社交多媒体数据的产生、分享
和获取都十分便利,然而这些数据都以碎片化的形式存在于社交媒体上,存在严重的碎片化,
搜索困难,浏览费时费力等问题。本文从照片和视频两个角度分别对社交多媒体的数据的关联表达
进行了研究。

对于照片集,本文提出了基于主题的照片集故事化表达系统——Monet。
系统首先对照片集进行总结,根据时间和位置信息将照片被划分到不同的事件,
再根据照片的视觉质量,所在事件的代表性,多样性选取一部分子集作为关键性照片。
其次,系统根据照片的内容选取合适的基于主题的编辑风格。
每张被选取的照片通过选取合适的相机运动被转换为视频片段。然后根据
电影编辑规则将一系列的视频特效,颜色过滤器,形状和转场效果附加到视频片段中。最终的视频
和音乐混流生成故事化的音乐视频。通过实验验证,
本文提出的系统达到了比当前最好的照片集时间检测和故事合成系统更好的效果。

对于视频,本文针对移动多摄像头视频提出了自动剪辑系统——MoVieUp,
同一事件中,多个摄像头从多个角度拍摄的时间上有重叠的一频剪辑成内容丰富,
能体现专业编辑水平的单一音视频流。我们通过用户调研总结了一系列可计算的视频编辑规则。
基于这些规则,系统通过音频指纹同步多摄像头视频,
评价音视频质量,在\emph{最小切换准则}下最大化音频质量生成音视剪辑结果。
对于视频剪辑,系统根据音频的节奏和语义信息检测视频切换点,
在切换点上基础上,系统在镜头运动一致性的约束下最大化视频质量和多样性,选取视频镜头。
实验评价结果显示我们的系统大的哦了比目前最好的视频自动剪辑系统更好的效果,提供了
更好的用户体验。

\section{研究工作展望}
社交多媒体数据的语义理解和关联表达是一个充满挑战的研究课题。
随着移动设备性能的提升和进一步普及,社交多媒体数据的规模和内容都将继续迅速增长,
相应的研究内容和应用场景也将更加复杂和多样,并受到越来越多的学术界和工业界的关注。
随着计算机视觉、机器学习和多媒体计算相关研究领域的发展,可以从一下几个方面对本文的
工作进行改进和扩展:

(1)语义理解方面。社交多媒体数据的语义理解既包含图片数据上的目标识别、目标检测、物体分割、
内容描述等研究内容,也包括视频数据上的场景识别,目标检测与跟踪,视频内容描述等。本文主要
解决照片数据的弱监督目标识别问题,利用大规模社交多媒体数据开展更多的弱监督语义理解研究
具有很大的研究意义和应用价值。例如,Bilen等人通过修改在标准数据集上预训练的深度卷积神经网络
同时学习区域选取和目标识别模型,实现了弱监督目标检测~\cite{Bilen_2016_CVPR}。
Pinheiro等人仅利用物体是否在照片中出现的弱监督信息在深度卷积神经网络上学习目标分割模型~\cite{pinheiro2014weakly}。
Rochan等人研究了在弱标注视频中的目标定位和分割问题~\cite{rochan2016weakly},训练数据仅需要提供
视频中出现的主要物体,算法就可以自动在每一帧中的定位出物体的位置并将物体与背景分开。
近年来,从视频生成文字描述获得了广泛的关注和研究,Shen等人提出了不需要准确标注的语句标注
的弱监督视频内容描述算法~\cite{shen2017weakly}。这些弱监督语义理解方法为社交多媒体数据
提供了更全面的分析和有益的借鉴。

(2)快速处理方面。虽然本文提出了快速有效的二阶在线特征选取算法用于社交多媒体数据的快速处理,
但当选取的特征数目较少时,算法的效果与批处理算法仍然有一定差距。如何从数据中挖掘更多的
信息,从而提高当前在线特征选取算法在特征数目较少时的效果仍然需要进一步的研究。
此外,如何自适应地选取特征数目也是需要解决的问题之一。
当前的特征选取算法需要手动指定特征的数目,而在实际应用中,算法应该能够不断地接收新的数据
并输出最紧凑最准确的模型。本文提出的特征选取算法主要解决单个特征的特征选取问题。当前,
研究人员探索了基于结构化信息的特征在文本问题上的应用,如文本分类~\cite{Wang2016TCH}和
文本聚类~\cite{Wang2015IWK}。如何将对快速有效地选取结构化特征并将他们应用到社交多媒体数据上
是富有挑战性的问题。
对于深度模型简化,本文提出的简化算法同样存在不能自适应地决定卷积核数目的问题。
此外,对于网络深度的简化仍需要进一步地探索。

(3)关联表达方面。
本文首先提出了基于主题的照片集故事话表达系统。
当前系统主要侧重照片集的内容从而进行故事化的表达,对于照片集的社交性没有做针对性的处理。
从社交网络中挖掘不同用户具有社交联系的照片集并针对不同用户生成个性化的故事化表达,从而
进一步的加强社交图片数据的关联表达。

对于视频方面的关联表达,当前提出的移动多摄像头视频自动剪辑系统基于内容上的相似性进行多样性
选取,为了利用更多可计算的视频编辑语法,需要确定移动摄像头拍摄的角度和距离~\cite{liu2012mm},
对视频做更深入的语义分析, 提高剪辑结果的多样性和专业性。
此外,当前系统在选取视频切换点时主要考虑音频的节奏和简单的语义信息,引入视频因素和更丰富
的音频语义可以提高切换点检测的效果仍然有待进一步的研究。
